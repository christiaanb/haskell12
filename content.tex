\section{Introduction}

Simulation is often used to inspect various characteristics of a system.
In the SoOS project\footnote{See \url{www.example.com} for details.} we
inspect service-oriented aspects of operating systems. One part of the
project requires to simulate the interactions between software
components of an operating system (in the following: OS) and between
OS and application software.

To be more elaborate: While creating of new operating system concepts
and regarding their interaction with the programmability of
large-scale systems, we found that existing simulation packages do not
seem to have the right abstractions for fast design exploration
\cite{cotson,omnet}. This was the motivation for the S(o)OS project
\cite{soos}.  It aims to research OS concepts and specific OS modules,
which aid in scalability of the complete software stack (both OS and
application) on future many-core systems.  One of the key concepts of
S(o)OS is that only those OS modules needed by a application thread,
are actually loaded into the (local) memory of a CPU core on which the
thread will run.  This execution environment differs from contemporary
operating systems where every core runs a complete copy of the
(monolithic) operating system.

The ability to simulate separate components of the OS and of the
application was the main goal to develop the OS simulator \soosim
\olcomment{cite the WATERS paper if it is accepted until
  submission}. \olcomment{would be too cool to be true: \soosim is
  also available from Hackage.\footnote{Issue \cd{cabal install
      soosim} to install the simulator. See also:
    \url{www.example.com/soosim}.}}
 The design decisions were
\begin{itemize}
\item to simulate all the communication between the components with
  message passing
\item to have global ``ticks'', designating time.
\end{itemize}
However, we required a way to implement blocking messages. In other
words, if a blocking message is sent, the sender is waiting for the
answer---regardless \emph{when} in the sender the message is sent. If
a blocking message is received, the reviever should suspend its
current work and process the message.\olcomment{is this true?} This
means, we required the notion of a suspending
computation---\emph{suspendable} at any moment of execution of the
simulated component. In the following we will show that using Haskell
\cite{haskell-report} enabled us to implement this property in a nice
way.

The contributions of this paper include
\begin{itemize}
\item a short overview of the SoOS project
\item the description of the eDSL used to implement \soosim 
\item the list of Haskell language features, we found particulary
  useful in this project.
\end{itemize}

The remaining part of this paper is organised as
follows. Section~\ref{sec:soos-project} gives an overview of the SoOS
project and of the role \soosim plays in the
project. Section~\ref{sec:soosim-an-overview} described the structure
of the \soosim simulator from the user's
perspective. Section~\ref{sec:dsl} gives an implementation overview of
the domain specific language, used to implement
\soosim. Section~\ref{sec:impl-detail} discusses Haskell features from
which our implementation especially
benefitted. Section~\ref{sec:related-work} covers related
work. Section~\ref{sec:concl-future-work} concludes and gives an
outlook for the further research.

\section{SoOS Project}
\label{sec:soos-project}

The SoOS project \olcomment{....... } \cite{soos}...

\section{\soosim: An Overview}
\label{sec:soosim-an-overview}

\paragraph{Basic structure.} \olcomment{few words on the ``building
  blocks''!}
The purpose of \soosim is to provide a platform that allows a
developer to observe the interactions between OS modules and
application threads.  For this reason the simulated hardware is highly
abstract.

In \soosim, the hardware platform is described as a set of nodes.
Each \emph{node} represents a physical computing object: such as a
core, complete CPU, memory controller, etc.  Every node has a local
memory of potentially infinite size.  The layout and connectivity
properties of the nodes are not part of the system description.  If
such a level of detail is required it would have to be modelled
explicitly by the user.

Each \emph{node} hosts a set of components.  A \emph{component}
represents an executable object: such as a thread, application, OS
module, etc.  Components communicate with each other either using
direct messaging, or through the local memory of a node.  Having both
explicit messaging and shared memories, \soosim supports the two well
known methods of communication.  Because multiple components can send
messages to one component, all component have a message queue.  All
components in a simulated system, even those hosted within the same
node, are executed concurrently.  The simulator poses no restrictions
as to which components can communicate with each other, nor to which
node's local memory they can read from and write to.  A user of
\soosim would have to model those restrictions explicitly if required.
A schematic overview of an example system can be seen in
Figure~\ref{fig:system}.

\def\svgwidth{\columnwidth}
\begin{figure}
%\includesvg{system}
\olcomment{graphics!!!}
\caption{System, abstracted.}
\label{fig:system}
\end{figure}

\paragraph{Agility.} Basic requirements that we would have towards
any simulator, include the facilities to straightforwardly simulate
the instantiation of application threads and OS modules.  Aside from
the fact that the S(o)OS-envisioned system will be dynamic as a result
of loading OS modules on-the-fly; large-scale systems also tend to be
dynamic in the sense that computing nodes can disappear (failure), or
appear (hot-swap).  Hence, we also require that our simulator
facilitates the straightforward creation and destruction of computing
elements.  Our current need for a simulator rests mostly in
formalizing the S(o)OS concept, and examining the interaction between
our envisioned OS modules and the application threads.  As such, being
able to extract highly accurate performance figures from a simulated
system is not a key requirement.  We do however wish to be able to
observe all interactions among application threads and OS modules.
Additionally, we wish to be able to \emph{zoom in} on particular
aspects of the behaviour of an application: such as memory access,
messaging, etc.

The simulator progresses all components concurrently in one discrete
step called a \hs{tick}.  During a \emph{tick}, the simulator passes
the content that is at the head of the message queue of each
individual component.  If the message queue of a component is empty, a
component will be executed with a \emph{null} message.  If desired, a
component can inform the simulator that it does not want to receive
these \emph{null} messages.  In that case the component will not be
executed by the simulator during a \emph{tick}.


\subsection{OS Component Descriptions}

The OS components are also specified in Haskell, each component is
modelled as a function. In case of \soosim, such a function is
executed within the context of the simulator, this means: in a
\emph{monad}.

Because the function is executed within the monad, it can have
\emph{side-effects} such as sending messages to other components, or
reading the memory of a local memory.  In addition, the function can
be temporarily suspended at (almost) any point in the code.  \soosim
needs to be able to suspend the execution of a function so that it may
emulate synchronous message passing between components, a subject we
will further elaborate later on.

We describe a component as a function that receives a user-defined
internal state as its first argument and a value of type \hs{SimEvent}
as its second argument.  The result of this function is the internal
state.  A value of type \hs{SimEvent} is either message from another
component, or a \emph{null} message.  We thus have the following type
signature for a component: 
%\numbersoff
\begin{code}
component :: State -> SimEvent -> SimM State
\end{code}

The simulator monad \hs{SimM} is described in
Figure~\ref{fig:code-simm}. \olcomment{Move that text block next to
  here?}  The user-defined internal \hs{State} can be used to store
any information that needs to perpetuate across simulator ticks.

To include a component description in the simulator, the developer
will have to create an instance of the \hs{ComponentIface} \emph{type class}.
% A \emph{type class} in Haskell can be compared to an interface
% definition as those known in object-oriented languages.  An
% \emph{instance} of a \emph{type class} is a concrete instantiation
% of such an interface.
%% haskell people know that!
The \hs{ComponentIface} requires the instantiation of the following values to completely define a component:

\begin{itemize}
  \item The initial internal state of the component.
  \item The unique name of the component.
  \item The monadic function describing the behaviour of the component.
\end{itemize}

Note, we are aiming at a high level of abstraction for the behavioural
descriptions of our OS modules, where the focus is mainly on the
interaction with other OS modules and application threads.


\subsection{Interaction with the simulator}

Components have several functions at their disposal to interact with
the simulator and consequently interact with other components.  The
available functions are:

\paragraph{\hs{createComponent}}
Instantiate a new component on a specified node. 
%; the component definition must be registered with the simulator.
\paragraph{\hs{invoke}}
Send a message to another component and wait for the answer.  Whenever a component uses this function it will be
temporarily suspended by the simulator.  Several simulator ticks
might pass before the % callee sends a
response.  Once the response is
% put in the message queue of the caller,
available the simulator resumes the execution of the calling
component.  Having this synchronization obviates the need to specify
the behaviour of a component as a finite state machine---a standard
approach in the area.
\paragraph{\hs{invokeAsync}}
Send a message to another component and register a handler with the
simulator to process the response.  In a contrast to \hs{invoke},
using this function will \emph{not} suspend the execution of the
component.
\paragraph{\hs{respond}}
Send a message to another component as a response to an invocation.
\paragraph{\hs{yield}}
Inform the simulator that the component does not want to receive
\emph{null} messages.
\paragraph{\hs{readMem}}
Read at a specified address of a node's local memory.
\paragraph{\hs{writeMem}}
Write a new value at a specified address of a node's local memory.
\paragraph{\hs{componentLookup}}
Lookup the unique identifier of a component on a specified node.
Components have two unique identifiers, their global \emph{name} (as
specified in the \hs{CompIface} instance), and a \hs{ComponentId} that
is a unique number corresponding to a specific instance of a
component.  When you want to \emph{invoke} a component, you need to
know the unique \hs{ComponentId} of the specific instance.  To give a
concrete example, using the system of Figure~\ref{img_system} as our
context: \emph{Thread (\#6)} wants to invoke the instance of the
\emph{Memory Manager} that is running on the same Node (\#2).  As
\emph{Thread (\#6)} was not involved with the instantiation of that OS
module, it has no idea what the specific \hs{ComponentId} of the
memory manager on Node \#2 is.  It knows the unique global name of all
memory managers, so it can use the \hs{componentLookup} function to
find the \hs{Memory Manager} with ID \#5 that is running on Node \#2.

\section{A Domain-specific Language for \soosim}
\label{sec:dsl}

% The DSL was embedded in Haskell \cite{haskell-report}.
For the sake of the implementation of ...... we embedded a DSL in
Haskell. \olcomment{FIX the explanation!}
Following the \emph{final tagless} \cite{final_tagless_embedding}
encoding of embedded languages in Haskell, we use a type class to
define the language constructs of our mini functional language with
mutable references.  A partial specification of the \hs{Symantics} (a
pun on \emph{syntax} and \emph{semantics}) type class, defining our
\emph{embedded language}, is shown in
Figure~\ref{fig:embedded_language_interface}.

\begin{figure}
\centering
\begin{code}
class Symantics repr where
  fun  :: (repr a -> repr b) -> repr (a :-> b)
  app  :: repr (a :-> b) -> repr a -> repr b
  ...
  drf   :: repr (Ref a) -> repr a
  (=:)  :: repr (Ref a) -> repr a -> repr Void
\end{code}
\caption{Embedded language, a partial definition.}
\label{fig:embedded_language_interface}
\end{figure}

\subsection{Final Tagless}

\olcomment{- Final tagless \cite{final_tagless_embedding}

- Tillmann embedded DSL with type classes \cite{Hofer:2008:PED:1449913.1449935} }

\section{Implementation Details}
\label{sec:impl-detail}

\olcomment{Rename to ``features we used''?}

\subsection{Type classes}

Type classes are naturally used for the eDSL encoding from the
previous section.  \olcomment{more details on this?}

Beyond this, we use type classes to express many
other useful abstractions. For instance, an \emph{interface} for the
component of the OS is a type class. Thus, each interacting ``building
block'' in the simulated software is an instance of that particular
type class. This enables great flexibility, including an option to
extend the kinds of simulated objects by third party.

\subsection{Monads}
\olcomment{Merge to OS Component Descriptions? Or: keep!}
We use a monad called \cd{SimM} to capture the \emph{State}
\olcomment{sure?} of the simulator. We use
\textsf{Control.Monad.Coroutine} to capture...

We use dynamic type for \olcomment{my god, what for?}

\begin{figure}
\centering
\begin{code*}
type SimMonad =  StateT SimState IO
data SimState = ...

newtype SimM a 
  = SimM { runSimM :: Coroutine 
      (RequestOrYield Unique Dynamic) 
      SimMonad a }
    deriving (Functor, Monad)

data RequestOrYield request response x
  = Request request (response -> x)
  | Yield   x

instance Functor (RequestOrYield x f) where
  fmap f (Request x g) = Request x (f . g)
  fmap f (Yield y)     = Yield (f y)
\end{code*}
\caption{Implementing \cd{SimM}.}
\label{fig:code-simm}
\end{figure}

The implementation of \cd{SimM}, sketched in
Figure~\ref{fig:code-simm}, enables us to reach the main
implementation goal: the ultimate suspension and resume of the
components upon message passing.  To suspend a computation we can now
merely write \cd{request componentId}, where the \cd{componentId} is
the unique ID of the OS component we are expecting a message from. The
execute a resumeable computation we issue \cd{resume computation}.

As one could infer from Figure~\ref{fig:code-simm}, the implementation
makes use of the coroutines in the following way. \olcomment{explain it!}


\section{Related Work}
\label{sec:related-work}

\olcomment{Related work from WATERS paper + more!}

\cite{house}
\cite{final_tagless_embedding,Hofer:2008:PED:1449913.1449935}

\section{Conclusions and Future Work}
\label{sec:concl-future-work}

\paragraph{Conclusions.} \olcomment{What have we seen?} We have
presented ....., and the techniques more commonly used in the
programming language research were highly applicable for our
purposes. We have demonstrated how the utilisation of final tagless
eDSL contruction \cite{...}, type classes \cite{...}, monads
\cite{...}, and coroutines \cite{...} facilitated an abstract and
concise implementation of an operating system simulator.

\paragraph{Real life simulations.} The major goal of the project is to
simulate the behaviour of a real life application and to draw
conclusions therefrom. This work is still ongoing.

\paragraph{Concurrency.} One aspect of the future work is the
concurrent \soosim. One option is to use Concurrent Haskell
\cite{ConcHs}, it provides (concurrent) green threads. However, it
would also require to use software transactional memory
\cite{springerlink:10.1007/s004460050028} because of multiple writes
to the same data. Examples of fruitful combinations of STM with other
concepts include
\cite{Harris:2008:CMT:1378704.1378725,Bieniusa:2010:BAA:1835698.1835714}.
Another option would be to use a \emph{parallel} Haskell like
Multicore Haskell \cite{marlow:rsm}, \cd{Par} monad \cite{par-monad}
or Eden \cite{eden}. The ``multiple writes'' issue, however, needs
some further handling in this case.

% \appendix
% \section{Appendix Title}
%
% This is the text of the appendix, if you need one.

\acks

%Acknowledgments, if needed.
This work was supported by the S(o)OS project, sponsored by the
European Commission under FP7-ICT-2009.8.1, Grant Agreement
No. 248465.

%%% Local Variables: 
%%% mode: latex
%%% TeX-master: "soosim"
%%% End: 
