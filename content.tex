\section{Introduction}


In the SoOS project\footnote{See <<project url>> for details.} we
inspect service-oriented aspects of operating systems. One part of the
project requires to simulate the interactions between software
components of an operating system (in the following: OS) and between
OS and application software.

The ability to simulate separate components of the OS and of the
application was the main goal to develop the OS simulator \soosim. The
design decisions were 
\begin{itemize}
\item to simulate all the communication between the components with
  message passing
\item to have global ``ticks'', designating time.
\end{itemize}
However, we required a way to implement blocking messages. In other
words, if a blocking message is sent, the sender is waiting for the
answer---regardless \emph{when} in the sender the message is sent. If
a blocking message is received, the reviever should suspend its
current work and process the message.\olcomment{is this true?} This
means, we required the notion of a suspending
computation---\emph{suspendable} at any moment of execution of the
simulated component. In the following we will show that using Haskell
enabled us to implement this property in a nice way.

The contributions of this paper include
\begin{itemize}
\item a short overview of the SoOS project
\item the description of the eDSL used to implement \soosim 
\item the list of Haskell language features, we found particulary
  useful in this project.
\end{itemize}

The remaining part of this paper is organised as
follows. Section~\ref{sec:soos-project} gives an overview of the SoOS
project and of the role \soosim plays in the
project. Section~\ref{sec:soosim-an-overview} described the structure
of the \soosim simulator from the user's
perspective. Section~\ref{sec:dsl} gives an implementation overview of
the domain specific language, used to implement
\soosim. Section~\ref{sec:impl-detail} discusses Haskell features from
which our implementation especially
benefitted. Section~\ref{sec:related-work} covers related
work. Section~\ref{sec:concl-future-work} concludes and gives an
outlook for the further research.

\section{SoOS Project}
\label{sec:soos-project}

\section{\soosim: An Overview}
\label{sec:soosim-an-overview}

\section{A Domain-specific Language for \soosim}
\label{sec:dsl}

\section{Implementation Details}
\label{sec:impl-detail}

\section{Related Work}
\label{sec:related-work}

\section{Conclusions and Future Work}
\label{sec:concl-future-work}

% \appendix
% \section{Appendix Title}
%
% This is the text of the appendix, if you need one.

\acks

%Acknowledgments, if needed.
This work was supported by EU FP7 SoOS project under grant ......
