%-----------------------------------------------------------------------------
%
%               Template for sigplanconf LaTeX Class
%
% Name:         sigplanconf-template.tex
%
% Purpose:      A template for sigplanconf.cls, which is a LaTeX 2e class
%               file for SIGPLAN conference proceedings.
%
% Guide:        Refer to "Author's Guide to the ACM SIGPLAN Class,"
%               sigplanconf-guide.pdf
%
% Author:       Paul C. Anagnostopoulos
%               Windfall Software
%               978 371-2316
%               paul@windfall.com
%
% Created:      15 February 2005
%
%-----------------------------------------------------------------------------


\documentclass[preprint]{sigplanconf}

% The following \documentclass options may be useful:
%
% 10pt          To set in 10-point type instead of 9-point.
% 11pt          To set in 11-point type instead of 9-point.
% authoryear    To obtain author/year citation style instead of numeric.

\usepackage{amsmath}

\usepackage{xspace}
\newcommand{\soosim}{SoOSiM\xspace}

\begin{document}

\conferenceinfo{Haskell '12}{date, City.} 
\copyrightyear{2012} 
\copyrightdata{[to be supplied]} 

%\titlebanner{banner above paper title}        % These are ignored unless
\preprintfooter{\today}   % 'preprint' option specified.

\title{Experience report: an OS similator in Haskell}
\subtitle{Subtitle Text, if any}

\authorinfo{Name1}
           {Affiliation1}
           {Email1}
\authorinfo{Name2\and Name3}
           {Affiliation2/3}
           {Email2/3}

\maketitle

\begin{abstract}
This is the text of the abstract.
\end{abstract}

\category{CR-number}{subcategory}{third-level}

\terms
term1, term2

\keywords
keyword1, keyword2

\section{Introduction}

In the SoOS project we inspect service-oriented aspects of operating
systems. One part of the project requires to simulate the interactions
between software components of an operating system (in the following:
OS) and between OS and application software.

The ability to simulate separate components of the OS and of the
application was the main goal to develop the OS simulator \soosim. The
design decisions were 
\begin{itemize}
\item to simulate all the communication between the components with
  message passing
\item to have global ``ticks'', designating time.
\end{itemize}
However, we required a way to implement blocking messages. In other
words, if a blocking message is sent, the sender is waiting for the
answer---regardless \emph{when} in the sender the message is sent. If
a blocking message is received, the reviever should suspend its
current work and process the message.\olcomment{is this true?} This
means, we required the notion of a suspending
computation---\emph{suspendable} at any moment of execution of the
simulated component. In the following we will show that using Haskell
enabled us to implement this property in a nice way.

The contributions of this paper include
\begin{itemize}
\item a short overview of the SoOS project
\item the description of the eDSL used to implement \soosim 
\item the list of Haskell language features, we found particulary
  useful in this project.
\end{itemize}

The remaining part of this paper is organised as follows. Section
.... gives an overview of the SoOS project and of the role \soosim
plays in the project. Section .... described the structure of the
\soosim simulator from the user's perspective. Section .... gives an
implementation overview of the domain specific language, used to
implement \soosim. Section .... discusses Haskell features from which
our implementation especially benefitted. Section .... covers related
work. Section .... concludes and gives an outlook for the further
research.

% \appendix
% \section{Appendix Title}
%
% This is the text of the appendix, if you need one.

\acks

%Acknowledgments, if needed.
This work was supported by EU FP7 SoOS project under grant ......

% We recommend abbrvnat bibliography style.

\bibliographystyle{abbrvnat}

% The bibliography should be embedded for final submission.
%% for now: use bibtex!

%% this is how you embedd a biblio
% \begin{thebibliography}{}
% \softraggedright
%
% \bibitem[Smith et~al.(2009)Smith, Jones]{smith02}
% P. Q. Smith, and X. Y. Jones. ...reference text...
%
% \end{thebibliography}

\end{document}
